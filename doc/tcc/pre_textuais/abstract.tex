% !TEX root = ../main.tex
\begin{resumo}[Abstract]
\begin{otherlanguage*}{english}
An empirical comparison of supervised machine-learning models is conducted to \textit{predict hourly wildfire occurrence} in Brazil’s Cerrado using climatic variables from INMET weather stations integrated with INPE’s BDQueimadas hotspots. A reproducible data pipeline was developed for automated extraction, harmonization, and spatio-temporal joining at the hourly scale, with explicit quality audits, standardization of sentinels, and scenario-based handling of missing data. Two data regimes are considered: (i) \emph{no-imputation} baselines and (ii) \emph{KNN-imputed} features for selected variables with substantive climatic relevance (e.g., solar radiation), enabling sensitivity analysis to the imputation choice.

The prediction task is framed as probabilistic binary classification. Five models are compared-logistic regression, Random Forest, XGBoost, linear SVM, and Gaussian Naive Bayes-configured to produce calibrated risk scores. Class imbalance is addressed through class weighting and threshold selection aligned with detection priorities. Model assessment employs spatio-temporal block validation and a chronological hold-out, reporting PR-AUC, recall, F1, precision–recall curves, Brier score, and calibration plots. Model interpretability is supported via SHAP analyses for tree-based methods to quantify the contribution of climatic drivers.

Results are discussed in terms of performance–operation trade-offs (sensitivity versus false alarms) and the measured impact of KNN imputation relative to no-imputation scenarios. The study contributes a transparent ETL and evaluation framework and offers practical guidance for risk-alert workflows based on INMET–INPE data within the Cerrado, with extensibility to other Brazilian biomes.
\textbf{Keywords:} wildfires; Cerrado; machine learning; climate; INMET; INPE.
\end{otherlanguage*}
\end{resumo}
