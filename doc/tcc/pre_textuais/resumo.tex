% !TEX root = ../main.tex
\begin{resumo}
Avalia-se, de forma comparativa, modelos de aprendizado de máquina para \textit{prever a ocorrência horária de focos de queimadas} no Cerrado a partir de variáveis climáticas de estações do INMET integradas aos focos do BDQueimadas do INPE. Define-se um pipeline reprodutível de extração, harmonização e junção espaço temporal, com auditoria de qualidade, padronização de sentinelas e tratamento de ausências em dois regimes: sem imputação e imputação por KNN em variáveis climaticamente relevantes.

O problema é formulado como classificação probabilística binária, com comparação entre regressão logística, Random Forest, XGBoost, SVM linear e Naive Bayes, produzindo escores calibráveis. O desbalanceamento é tratado com pesos de classe e ajuste de limiar. A avaliação utiliza validação por blocos espaço temporais e hold out cronológico, reportando PR AUC, recall, F1, curvas de precisão versus revocação, Brier score e calibração. Os resultados são discutidos em termos de sensibilidade e alarmes falsos, bem como do impacto da imputação KNN em relação ao cenário sem imputação, oferecendo diretrizes para fluxos de alerta baseados em dados INMET e INPE.

\textbf{Palavras-chave}: queimadas; Cerrado; previsão; clima; aprendizado de máquina; INMET; INPE.
\end{resumo}
