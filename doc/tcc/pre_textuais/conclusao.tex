% !TEX root = ./main.tex

\chapter{Conclusão}

Este trabalho apresentou uma análise comparativa de algoritmos de aprendizado de máquina aplicados à previsão de queimadas no bioma Cerrado, a partir da integração de dados meteorológicos do INMET e focos de calor do BD Queimadas (INPE). Foram comparados modelos Random Forest, XGBoost, SVM e MLP, com ênfase em desempenho, interpretabilidade e robustez a desbalanceamento.

Os resultados indicam que os métodos baseados em árvores, especialmente o XGBoost, oferecem melhor equilíbrio entre acurácia e interpretabilidade, possibilitando a identificação de variáveis climáticas mais determinantes no processo de ignição. O uso de técnicas de explicabilidade (SHAP) reforçou a transparência dos modelos e sua aplicabilidade operacional em contextos ambientais.

Como trabalhos futuros, propõe-se a expansão do estudo para outros biomas brasileiros e a incorporação de variáveis de uso do solo e combustível, bem como a avaliação de arquiteturas híbridas com dados de satélite de alta resolução.
