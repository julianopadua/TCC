% !TEX root = ./main.tex

\chapter{Introdução}

As queimadas afetam ciclicamente a biodiversidade, a saúde pública e a economia. A literatura indica avanços significativos com o uso de aprendizado de máquina (ML) na previsão do fogo. Modelos Random Forest e SVM alcançaram acurácias superiores a 85\,\% em cenários do Paquistão~\cite{shahzad2024pakistan} e da China~\cite{pang2022china}, mesmo quando treinados apenas com dados climáticos diários e índices de vegetação. Em escala global, abordagens que incorporam combustível e fontes de ignição reduziram drasticamente falsos alarmes~\cite{giuseppe2025global}. Revisões recentes mostram que arquiteturas profundas, como CNNs e ConvLSTM, obtêm melhor desempenho na propagação do fogo, mas enfatizam a necessidade de dados de alta qualidade para treinamento~\cite{andrianarivony2024review}. No contexto brasileiro, um estudo conduzido na APA Triunfo do Xingu aplicou Random Forest e XGBoost para mapear suscetibilidade a incêndios, comprovando a relevância de variáveis climáticas e de pressão antrópica~\cite{freitas2025xingu}.

\section{Motivação}

No bioma Cerrado, entre 2001 e 2023, mais de 370\,mil focos de calor foram detectados apenas pelos sensores MODIS. A previsão robusta de queimadas diárias pode reduzir custos de combate, mitigar emissões de CO\textsubscript{2} e preservar serviços ecossistêmicos. Apesar de avanços internacionais, poucos trabalhos nacionais integram dados meteorológicos de alta resolução com modelos recentes de ML; o estudo de Freitas et~al.~\cite{freitas2025xingu} é uma das raras exceções. Esse contexto motiva a presente pesquisa, que busca sistematizar, avaliar e comparar técnicas de ML capazes de antecipar a ignição com antecedência suficiente para suporte à tomada de decisão.

\section{Objetivos}

\subsection{Objetivo Geral}
Comparar o desempenho e a interpretabilidade de diferentes algoritmos de aprendizado de máquina na previsão diária de focos de queimadas no Cerrado, utilizando variáveis climáticas provenientes do INMET integradas aos registros do BD Queimadas (INPE).

\subsection{Objetivos Específicos}
\begin{itemize}
    \item Construir um \textit{dataset} diário (2005--2025) combinando variáveis meteorológicas e focos de calor georreferenciados;
    \item Treinar e avaliar modelos Random Forest~\cite{breiman2001randomForest}, XGBoost~\cite{chen2016xgboost}, SVM~\cite{cortes1995svm} e redes neurais multicamadas~\cite{rumelhart1986backprop};
    \item Investigar a influência de grupos de variáveis (temperatura, umidade, precipitação, vento, índices de vegetação);
    \item Aplicar técnicas de explicabilidade (SHAP) para identificar os fatores climáticos mais determinantes;
    \item Propor diretrizes para implementação de um sistema de alerta operacional.
\end{itemize}
