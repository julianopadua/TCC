% !TEX root = ../main.tex

\chapter{Introdução}

\section{Contexto e motivação}
As queimadas no Cerrado configuram um fenômeno recorrente e estruturado no tempo e no espaço, com sazonalidade marcada na estação seca e associação a tipos de cobertura e vetores antrópicos \cite{nascimento2011analise}. A literatura recente indica que abordagens baseadas em dados são adequadas para capturar tais padrões multivariados e não lineares, incluindo modelos clássicos e arquiteturas modernas para previsão de ocorrência e propagação \cite{andrianarivony2024review,freitas2025xingu}. Para que estimativas sejam úteis em apoio à decisão, a construção de uma base integrada é requisito metodológico: integração entre focos do BDQueimadas e variáveis climáticas do INMET em escala horária, com controle explícito de qualidade, padronização de formatos e chaves espaço temporais. Em séries climáticas, dados faltantes são frequentes e afetam estatísticas e previsores; revisões e estudos aplicados recomendam tratar a imputação como escolha adicional de modelagem, justificando cenários comparativos com e sem preenchimento \cite{alejosanchez2025review,afridayamoah2020imputation}. Métodos simples e transparentes, como KNN, apresentam desempenho competitivo em lacunas leves e são apropriados como baseline de imputação em pipelines reprodutíveis \cite{chehal2023imputation}. Nesse quadro, comparar famílias de modelos sob protocolos de validação adequados torna se necessário, pois não há preferência a priori entre algoritmos sem hipóteses adicionais sobre a distribuição geradora \cite{wolpert1996lack}.

\section{Objetivos}
Objetivo geral: avaliar, de forma comparativa, modelos de aprendizado de máquina para previsão horária da ocorrência de focos de queimadas no Cerrado, a partir da integração INMET–BDQueimadas e de um pipeline reprodutível de dados.

Objetivos específicos:
\begin{itemize}
    \item construir base horária integrada INMET–BDQueimadas com auditoria de qualidade, padronização de sentinelas e versionamento por cenários;
    \item definir cenários de dados sem imputação e com imputação por KNN em variáveis climaticamente relevantes, permitindo análise de sensibilidade;
    \item formular o problema como classificação probabilística binária e comparar regressão logística, Random Forest, XGBoost, SVM linear e Naive Bayes \cite{breiman2001randomForest,chen2016xgboost};
    \item empregar validação por blocos espaço temporais e \textit{hold out} cronológico, reportando PR AUC, recall, F1, Brier score e calibração \cite{andrianarivony2024review};
    \item analisar interpretabilidade em modelos de árvores por meio de SHAP, quantificando a contribuição de variáveis climáticas para o risco estimado;
    \item discutir compromissos entre desempenho e custo operacional, oferecendo diretrizes para fluxos de alerta baseados em INMET e INPE.
\end{itemize}

\section{Organização}
O texto organiza se como segue. O Capítulo 2 apresenta a fundamentação teórica em Inteligência Artificial e a adequação do problema de previsão de focos à classificação probabilística, com definição dos modelos comparados. O Capítulo 3 descreve o pipeline de dados, incluindo integração INMET–BDQueimadas, auditoria de qualidade e a estratégia de cenários com e sem imputação por KNN \cite{alejosanchez2025review,afridayamoah2020imputation,chehal2023imputation}. O Capítulo 4 detalha o desenho experimental de comparação, protocolos de validação e métricas, à luz da literatura comparativa e dos resultados No Free Lunch \cite{martinovic2025comparative,wolpert1996lack}. O Capítulo 5 apresenta resultados e análises de interpretabilidade, seguido por discussão e limitações no Capítulo 6. O Capítulo 7 traz conclusões e possibilidades de extensão para outros biomas e aprimoramentos do pipeline.
