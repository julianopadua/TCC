% !TEX root = ../main.tex

\chapter{Metodologia}
\label{cap:metodologia}

\section{Obtenção dos dados}
\label{sec:obtencao-dados}

Para a composição do conjunto de dados utilizado nos experimentos, integraram-se bases de sensoriamento remoto e séries meteorológicas de superfície. Os dados de focos de calor provêm do sistema BDQueimadas, do Instituto Nacional de Pesquisas Espaciais (INPE), especificamente da coleção \textit{Brasil\_sat\_ref}. Esta fonte fornece variáveis fundamentais como o \textit{Fire Radiative Power} (FRP) e o Risco de Fogo, além de identificadores únicos e coordenadas geográficas obtidas via \textit{scraping} do \textit{dataserver} COIDS\footnote{\url{https://github.com/julianopadua/TCC}}. 

As variáveis climáticas foram extraídas das estações automáticas do Instituto Nacional de Meteorologia (INMET). A coleta foi automatizada por scripts de \textit{scraping} que padronizam séries horárias de temperatura, umidade, pressão atmosférica, radiação e vento. Para assegurar a reprodutibilidade, os dados brutos são armazenados em uma estrutura hierárquica em \texttt{data/raw/}, enquanto as versões tratadas e harmonizadas seguem para \texttt{data/processed/}.

\section{Processamento e Integração (Pipeline de ETL)}
\label{sec:pipeline-extracao}

O fluxo de \textit{Extract-Transform-Load} (ETL) foi projetado para correlacionar eventos discretos de fogo com registros climáticos contínuos. A integração ocorre mediante a criação de uma chave sintética que combina o selo temporal (arredondado para a hora cheia), o município e a unidade da federação, após normalização de caracteres e conversão para ASCII.

A Figura~\ref{fig:pipeline-dataset} ilustra o fluxo de dados, detalhando a transição entre a ingestão das fontes brutas e a geração das bases consolidadas em formato \texttt{.parquet}.

\begin{figure}[H]
    \centering
    \includegraphics[width=\textwidth]{imagens/dataset_pipeline.png}
    \caption{Fluxo geral do pipeline de dados integrando BDQueimadas e INMET.}
    \label{fig:pipeline-dataset}
\end{figure}

O módulo \path{build_dataset.py} realiza o alinhamento espaço-temporal para o bioma Cerrado, definindo a variável alvo $y \in \{0, 1\}$. Atribui-se o rótulo positivo (\texttt{HAS\_FOCO} = 1) aos registros onde houve ao menos uma detecção de queimada no intervalo e localização correspondentes; caso contrário, o registro é classificado como negativo.

\section{Tratamento de dados ausentes e cenários}
\label{sec:bases-modelagem}

A qualidade da base integrada foi avaliada mediante auditoria de dados faltantes, identificando-se códigos sentinela (\texttt{-999} e \texttt{-9999}) e valores nulos. Para mensurar o impacto dessas ausências no desempenho preditivo, os experimentos foram estruturados em seis cenários distintos, variando entre a exclusão seletiva de variáveis (\textit{base\_A}), a remoção de instâncias incompletas (\textit{base\_C}, \textit{base\_D}) e a imputação por \textit{K-Nearest Neighbors} (KNN) (\textit{base\_B}, \textit{base\_E}).

A imputação via KNN foi adotada para preencher lacunas em variáveis meteorológicas aproveitando a correlação local, sendo aplicada sobre bases com e sem a variável de radiação global. Este procedimento visa mitigar o viés introduzido por falhas instrumentais nas estações automáticas, permitindo uma comparação robusta entre os modelos treinados em diferentes regimes de completude de dados.

\section{Procedimentos experimentais}
\label{sec:protocolo-experimentos}

A tarefa de predição é formulada como uma classificação binária em um cenário de alto desbalanceamento de classes. Para garantir uma avaliação justa e evitar o sobreajuste temporal (\textit{data leakage}), utilizou-se a técnica de \textit{time-series split}, onde o conjunto de teste é cronologicamente posterior ao de treinamento.

Os algoritmos avaliados — Regressão Logística, Naive Bayes, SVM, Random Forest e XGBoost — foram testados sob métricas sensíveis à detecção de eventos raros. Além da curva ROC, priorizaram-se a \textit{Precision-Recall AUC} e o \textit{F1-Score}. A calibração probabilística dos modelos foi mensurada pelo \textit{Brier Score}, assegurando que as estimativas de risco fornecidas pelo sistema sejam representativas da frequência observada de queimadas no Cerrado.