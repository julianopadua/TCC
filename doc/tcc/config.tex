% !TEX root = main.tex
\documentclass[12pt,openright,twoside,a4paper]{abntex2}

% --------------------------------------------------
% Fonte principal: Helvetica (parecida com Arial)
% --------------------------------------------------
\usepackage[scaled=0.92]{helvet}
\renewcommand{\familydefault}{\sfdefault}

% Matemática e Fontes
\usepackage{newtxmath}
\usepackage{textcase} % Essencial para conversão de maiúsculas com acento

% Margens padrão da UFSCar
\usepackage[top=3cm,bottom=2cm,left=3cm,right=2cm]{geometry}

% Pacotes essenciais
\usepackage[utf8]{inputenc}
\usepackage[T1]{fontenc}
\usepackage[brazil]{babel}
\usepackage{graphicx}
\usepackage{amsmath}
\usepackage{setspace}
\usepackage{fancyhdr} % Para controle total dos cabeçalhos
\usepackage{ragged2e}
\usepackage{indentfirst}
\usepackage[alf]{abntex2cite}
\usepackage{booktabs}
\usepackage{enumitem}
\usepackage{float}
\usepackage{multirow}
\usepackage{caption}
\usepackage{subcaption}
\usepackage{tikz}
\usepackage{listings}

% Config básica pros códigos
\lstset{
  basicstyle=\ttfamily\footnotesize,
  breaklines=true,
  breakatwhitespace=true,
  columns=fullflexible
}

% Hiperlinks
\usepackage{hyperref}
\hypersetup{
  colorlinks=true,
  linkcolor=black,
  citecolor=black,
  urlcolor=black
}

% --------------------------------------------------
% CONFIGURAÇÃO DE CABEÇALHOS (ESTILO ARCENCIO)
% --------------------------------------------------
% Define o estilo 'fancy' para páginas textuais
\pagestyle{fancy}
\fancyhf{} % Limpa todas as configurações

% Redefine a forma como o capítulo e a seção aparecem no cabeçalho
\renewcommand{\chaptermark}[1]{\markboth{#1}{}}
\renewcommand{\sectionmark}[1]{\markright{#1}}

% Página Par (Even - Esquerda): Página à esquerda, Capítulo à direita
\fancyhead[LE]{\thepage}
\fancyhead[RE]{\textit{Capítulo \thechapter. \leftmark}}

% Página Ímpar (Odd - Direita): Seção à esquerda, Página à direita
\fancyhead[LO]{\textit{\thesection. \rightmark}}
\fancyhead[RO]{\thepage}

\renewcommand{\headrulewidth}{0.4pt} % Linha horizontal

% Garante que páginas de início de capítulo fiquem sem cabeçalho (vazias)
\fancypagestyle{plain}{
  \fancyhf{}
  \renewcommand{\headrulewidth}{0pt}
}
\aliaspagestyle{chapter}{plain}

% Ajuste de espaçamento de listas para ser mais compacto (como o exemplo)
\setlist{nosep} 

% --------------------------------------------------
% INFORMAÇÕES INSTITUCIONAIS
% --------------------------------------------------
\titulo{Análise Comparativa do Desempenho de Modelos de Machine Learning na Previsão de Focos de Incêndio no Cerrado Utilizando Variáveis Climáticas}
\autor{Juliano Eleno Silva Pádua}
\local{São Carlos - SP}
\data{2025}
\instituicao{
  Universidade Federal de São Carlos\\
  Centro de Ciências Exatas e de Tecnologia -- CCET\\
  Departamento de Computação -- DC\\
  Curso de Bacharelado em Engenharia de Computação
}
\orientador{Prof. Dr. Alexandre Levada}
\tipotrabalho{Trabalho de Conclusão de Curso}

% ===== Capa (Corrigindo PáDUA) =====
\renewcommand{\imprimircapa}{
  \begin{center}
    {\ABNTEXchapterfont\large\MakeTextUppercase{\imprimirinstituicao}}
    \vspace*{4cm}

    {\ABNTEXchapterfont\bfseries\LARGE\imprimirtitulo}
    \vspace*{5cm}

    % Força a conversão correta de Juliano Eleno Silva Pádua para maiúsculas
    {\large\MakeTextUppercase{\imprimirautor}}
    \vspace*{6cm}

    {\large\imprimirlocal}\\
    {\large\imprimirdata}
  \end{center}
}

% ===== Folha de rosto =====
\makeatletter
\renewcommand{\folhaderostocontent}{
  \begin{center}
    {\large\MakeTextUppercase{\imprimirautor}}

    \vspace*{6cm}
    {\ABNTEXchapterfont\bfseries\Large \imprimirtitulo \par}

    \vspace*{3cm}
  \end{center}

  \noindent\hfill
  \begin{minipage}{0.52\textwidth}
    \justifying
    \SingleSpacing
    \noindent Trabalho de Conclusão de Curso apresentado ao curso de Engenharia de Computação da Universidade Federal de São Carlos, como requisito parcial para a obtenção do título de Bacharel em Engenharia de Computação.

    \vspace{1em}
    \noindent Orientação: \imprimirorientador
  \end{minipage}

  \vfill

  \begin{center}
    {\large \imprimirlocal}\\[2pt]
    {\large \imprimirdata}
  \end{center}
}
\makeatother