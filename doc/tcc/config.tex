% !TEX root = main.tex
\documentclass[12pt,openright,twoside,a4paper]{abntex2}

% --------------------------------------------------
% Fonte principal: Helvetica (parecida com Arial)
% (funciona com pdfLaTeX, não precisa trocar compilador)
% --------------------------------------------------
\usepackage[scaled=0.92]{helvet} % ajusta o tamanho pra ficar mais elegante
\renewcommand{\familydefault}{\sfdefault}

% Matemática mais "padrão" (Times-like)
\usepackage{newtxmath}

% Margens padrão da UFSCar
\usepackage[top=3cm,bottom=2cm,left=3cm,right=2cm]{geometry}

% Pacotes essenciais
\usepackage[utf8]{inputenc}
\usepackage[T1]{fontenc}
\usepackage[brazil]{babel}
\usepackage{graphicx}
\usepackage{amsmath}
\usepackage{setspace}
\usepackage{fancyhdr}
\usepackage{ragged2e}
\usepackage{indentfirst}
\usepackage[alf]{abntex2cite}
\usepackage{booktabs}
\usepackage{enumitem}
\usepackage{float}
\usepackage{multirow}
\usepackage{caption}
\usepackage{subcaption}
\usepackage{tikz}
\usepackage{listings}

% Config básica pros códigos não ficarem gigantes e quebrarem linha
\lstset{
  basicstyle=\ttfamily\footnotesize,
  breaklines=true,
  breakatwhitespace=true,
  columns=fullflexible
}

% Hiperlinks
\usepackage{hyperref}
\hypersetup{
  colorlinks=true,
  linkcolor=black,
  citecolor=black,
  urlcolor=black
}

% Informações institucionais (para capa e folha de rosto)
\titulo{Análise Comparativa do Desempenho de Modelos de Machine Learning na Previsão de Focos de Incêndio no Cerrado Utilizando Variáveis Climáticas}
\autor{Juliano Eleno Silva Pádua}
\local{São Carlos - SP}
\data{2025}
\instituicao{
  Universidade Federal de São Carlos\\
  Centro de Ciências Exatas e de Tecnologia -- CCET\\
  Departamento de Computação -- DC\\
  Curso de Bacharelado em Engenharia de Computação
}
\orientador{Prof. Dr. Alexandre Levada}
\tipotrabalho{Trabalho de Conclusão de Curso}

% ===== Capa (modelo UFSCar) =====
\renewcommand{\imprimircapa}{
  \begin{center}
    {\ABNTEXchapterfont\large\MakeUppercase\imprimirinstituicao}
    \vspace*{4cm}

    {\ABNTEXchapterfont\bfseries\LARGE\imprimirtitulo}
    \vspace*{5cm}

    {\large\imprimirautor}
    \vspace*{6cm}

    {\large\imprimirlocal}\\
    {\large\imprimirdata}
  \end{center}
}

% ===== Folha de rosto (modelo UFSCar) =====
\makeatletter
\renewcommand{\folhaderostocontent}{
  \begin{center}
    % Autor no topo
    {\large \imprimirautor}

    \vspace*{6cm}
    % Título centralizado
    {\ABNTEXchapterfont\bfseries\Large \imprimirtitulo \par}

    \vspace*{3cm}
  \end{center}

  % Bloco descritivo justificado à direita
  \noindent\hfill
  \begin{minipage}{0.52\textwidth}
    \justifying
    \SingleSpacing
    \noindent Trabalho de Conclusão de Curso apresentado ao curso de Engenharia de Computação da Universidade Federal de São Carlos, como requisito parcial para a obtenção do título de Bacharel em Engenharia de Computação.

    \vspace{1em}
    \noindent Orientação: \imprimirorientador
  \end{minipage}


  \vfill

  % Cidade/UF e ano no rodapé
  \begin{center}
    {\large \imprimirlocal}\\[2pt]
    {\large \imprimirdata}
  \end{center}
}
\makeatother
